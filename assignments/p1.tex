\documentclass[titlepage,12pt]{book}
\usepackage[utf8]{inputenc}

\usepackage[margin=1in]{geometry}
\usepackage[colorlinks,linkcolor=blue,urlcolor=blue]{hyperref}
\usepackage{ntheorem}
\usepackage{enumitem}
\usepackage{amsmath}
\usepackage{amssymb}
\usepackage{tikz}
\usepackage{mathtools}


% configure ntheorem
\theoremstyle{break}
\theorembodyfont{\upshape} % no italics in body font
\newtheorem{exercise}{Exercise}
\newtheorem{solution}{Solution}
\newtheorem{lemm}{Lemma}
\newtheorem{corol}{Colorally}
\newtheorem*{theorem}{Theorem}
\newtheorem*{proof}{Proof}

%configure mathtools
\DeclarePairedDelimiter\ceil{\lceil}{\rceil}
\DeclarePairedDelimiter\floor{\lfloor}{\rfloor}

% shortcuts
\newcommand{\enum}[1]{\begin{enumerate}[label=(\alph*)] #1 \end{enumerate}}
\newcommand{\enumr}[1]{\begin{enumerate}[label=(\roman*)] #1 \end{enumerate}}

\newcommand{\TODO}{\textbf{TODO }}
\newcommand{\heart}{\ensuremath\heartsuit}
\newcommand{\blank}{\makebox[1cm]{\hrulefill}}
\newcommand{\blankk}{\blank\space}

% math shortcuts
\newcommand*\closure[1]{\overline{#1}}
\newcommand{\abs}[1]{\left|#1\right|}



\begin{document}

\mainmatter

\begin{solution}
    In a regular tetrahedron, all faces are the same size and shape (congruent) and all edges are the same length.
    Consider vertices of only two intersecting faces and let its coordinates of the vertices be:

    Face 1/ Plane 1: $(0,0,0), (1, 1, 0), (1, 0, 1)$

    Face 2/ Plane 2: $(0,0,0), (1, 1, 0), (0, 1, 1)$
    

    Now, in Plane 1:
    
    $n1 = (1, 1, 0) \times (1, 0, 1) = (1, -1, -1)$


    In Plane 2:

    $n2 = (1, 1, 0) \times (0, 1, 1) = (1, -1, 1)$

    The dihedral angle is:

    $|n1||n2| \cos(\theta) = n1 \cdot n2$
    
    $(\sqrt{3}) (\sqrt{3}) \cos(\theta) =  1 + -1 + -1$

    $\theta = \arccos(-1/3)$
\end{solution}


\begin{solution}
    \enum {
        \item $|u + v|^{2} = (u + v) \cdot (u + v) = u \cdot (u + v) + v \cdot (u + v) = u \cdot u + u \cdot v + v \cdot u + v \cdot v = |u|^{2} + |v|^{2} + 2 (u \cdot v)$
        Similary, $|u - v|^{2} = |u|^{2} + |v|^{2} - 2 (u \cdot v)$
    
        Now, $|u + v|^{2} - |u - v|^{2} = 4 (u \cdot v)$.
        Thus, $1/4 (|u + v|^{2} - |u - v|^{2}) = (u \cdot v)$.

        \item $(u + v) / (|u + v|)$
    }
\end{solution}

\end{document}

%%% Local Variables:
%%% mode: latex
%%% TeX-master: t
%%% End:
