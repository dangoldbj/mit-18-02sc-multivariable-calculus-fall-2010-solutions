\documentclass[titlepage,12pt]{book}
\usepackage[utf8]{inputenc}

\include{../preamble}

\begin{document}

\mainmatter

\begin{solution}
    \enum {
        \item I won't draw a solution here, but will provide an explanation.
        Picture a triangle. If you pictured it in 2 dimensions then try picturing it again in three dimensions.
        Now replace the sides of the triangle which you probably visualized as lines to infinite planes.
        Notice how each of these planes intersect at the vertices of previously visualized triangle which are now lines in 3d.

        Also notice how these lines are distinct from each other yet they do not intersect hence parallel.

        \item The normal vectors $n1, n2, n3$ are also normal to lines $l1, l2, l3$ respectively.
            Since $l1 // l2 // l3$ we can cut through them in a single plane hence the vectors $n1, n2, n3$ all lie on the same plane.
        
        \item Equation of plane $i$ is $x \cdot n_{i} = a_{i}$. But since the planes do not intersect. There is no solution.
        Hence, $det(A) = 0$. This implies $n_{1} \cdot n_{2} \times n_{3} = 0$.
    }
\end{solution}

\begin{solution}
    solution
    \enum {
        \item $A = \begin{bmatrix}
            1 & 2 & 3\\
            1 & 3 & 5\\
            3 & 5 & 8
            \end{bmatrix}$

            $Ax = P$
    }
\end{solution}

\end{document}

%%% Local Variables:
%%% mode: latex
%%% TeX-master: t
%%% End:
